% Created 2016-09-20 Tue 07:50
\documentclass[presentation, smaller]{beamer}
\usepackage[latin1]{inputenc}
\usepackage[T1]{fontenc}
\usepackage{fixltx2e}
\usepackage{graphicx}
\usepackage{longtable}
\usepackage{float}
\usepackage{wrapfig}
\usepackage{rotating}
\usepackage[normalem]{ulem}
\usepackage{amsmath}
\usepackage{textcomp}
\usepackage{marvosym}
\usepackage{wasysym}
\usepackage{amssymb}
\usepackage{hyperref}
\tolerance=1000
\mode<beamer>{\usetheme{Boadilla}\usecolortheme[RGB={40,100,30}]{structure}}
\usebackgroundtemplate{\includegraphics[width=\paperwidth]{MNRwhite}}
\setbeamersize{text margin left=10mm}
\usetheme{default}
\author{Proof of Concept}
\date{September 20, 2016.}
\title{FishNet-Portal}
\hypersetup{
  pdfkeywords={},
  pdfsubject={},
  pdfcreator={Emacs 24.5.1 (Org mode 8.2.10)}}
\begin{document}

\maketitle


\begin{frame}[label=sec-1]{FishNet (FN)-Portal}
\begin{itemize}
\item make data housed in UGLMU databases accessible in your browser
\item Project Tracker:
\begin{itemize}
\item project documentation (reports, protocols and
associated milestones)
\end{itemize}
\item FN\_Portal:
\begin{itemize}
\item Net set, catch count and biological data collected in the
project
\end{itemize}
\item complementary back-end to project tracker
\end{itemize}
\end{frame}

\begin{frame}[label=sec-2]{Status}
\begin{itemize}
\item currently proof of concept/prototype
\item suggestion welcome
\end{itemize}
\end{frame}

\begin{frame}[label=sec-3]{Implementation Stack:}
\begin{itemize}
\item leaflet for maps
\item crossfilter.js data filtering
\item dc.js (D3) plots
\end{itemize}
\end{frame}

\begin{frame}[label=sec-4]{Views}
\begin{itemize}
\item project detail
\item species detail
\item trend-through time
\end{itemize}
\end{frame}

\begin{frame}[label=sec-5]{Projects}
\begin{itemize}
\item currently includes data from:
\begin{itemize}
\item Offshore master
\item Nearshore master
\item Smallfish master
\end{itemize}
\end{itemize}
\end{frame}

\begin{frame}[label=sec-6]{Caveats}
\begin{itemize}
\item general information
\item basic exploratory analysis
\item limited to common fields/attributes
\item sophisticated, specialized analysis out-of-scope
\end{itemize}
\end{frame}

\begin{frame}[label=sec-7]{Next steps}
\begin{itemize}
\item extended to:
\begin{itemize}
\item other project types
\item Lake Superior
\end{itemize}
\item formatting
\item add missing filters
\item Catches/Samples by ROI
\end{itemize}
\end{frame}
% Emacs 24.5.1 (Org mode 8.2.10)
\end{document}
